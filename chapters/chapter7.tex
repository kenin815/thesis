\section{結論}
本論文主要目的為不經過語音辨識系統達成口述詞彙偵測的目的。傳統的口述詞彙偵測系統的檢索效果好壞完全依靠著前面的語音辨識系統的辨識效果為何,若語音辨識系統錯誤率偏高,會嚴重影響檢索系統。且語音訊號經過了辨識系統,許多語音中珍貴的資訊如韻律、語速和語者特徵等就消失了。藉由類神經網路直接由語音訊號進行搜索,希望能夠避免掉語音辨識系統的影響。

第\ref{ch3}章中,提出了一個基本的模型架構,利用遞迴類神經網路將查詢詞聲學序列跟語音文件序列編碼成為一向量,藉由類神經網路判別查詢詞是否出現在語音文件中,雖然平均準確率輸給了動態時間規劃,但此模型已為捨棄掉語音辨識的檢索系統。

第\ref{ch4}章中,將第\ref{ch3}章的模型利用專注式機制改良,使其在編碼器產生語音文件的向量時藉由利用專注式機制,將多餘且跟查詢詞無關的資訊濾除,使其能夠專注在疑似查詢詞的位置。在此章中,將專注式權重視覺化後,發現在查詢詞位置的結束處,專注式權重都特別大,證明了模型有關注在查詢詞出現的位置,使其模型效能大幅提升。最後同時利用模型跟動態時間規劃,能夠使檢索效能再往上提升。

第\ref{ch5}章中,引入語音詞向量的概念,將整段語音文件去訓練語音詞向量,希望產生出的語音文件向量也能夠有語音詞向量的特性。而藉由查詢詞跟文件的語音詞向量的歐式距離,決定查詢詞是否出現在文件當中。將語音詞向量視覺化後,證明了文件跟查詢詞的歐式距離並無絕對的關係,且文件跟查詢詞的查詢詞有著不同的意義。在查詢詞中,能夠將發音類似的結尾分在同個區域。而在文件中,無法找出特別的分類關係。

在最後的第\ref{ch6}章中,將前幾章的模型做結合希望能夠有所進步,而實驗證明同時訓練語音詞向量並進行口述詞彙偵測,會使模型的效能變差一點,將語音詞向量當作正規化並無法幫助模型在口述詞彙偵測的表現。
\section{未來與展望}
首先,本篇論文其關鍵的部分在於專注式權重的計算,未來可以嘗試更多的專注式權重的算法,如由類神經網路產生出專注式權重或者將每個時間的編碼器產生的查詢詞向量跟語音文件向量都去產生一個專注式權重,進而改善模型的效能。

再者,本篇論文皆在討論查詢詞為語音的情況,未來也可以朝向查詢詞為人為輸入的文字,而直接由文字跟語音找尋其相似特徵,難度會比較高。因文字跟語音訊號本身有巨大的差異,要如何將語音和文字的配對是一個要思考的大問題。

最後,未來也可以朝向語意檢索的方向邁進,同樣不經過語音辨識,但能夠將查詢詞相關的文件也檢索出來。若能夠將語義的資訊放入編碼的向量中,則可利用查詢詞擴展,將查詢詞附近的向量也進行檢索,進而找出語意相關的文件。
