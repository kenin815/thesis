\NTUtitlepage  % 產生論文封面

\newpage
\setcounter{page}{1}
\pagenumbering{roman}

\NTUoralpage  % 產生口試委員會審定書

\mydoublespacing
\begin{acknowledgement} %誌謝
轉眼之間,碩士兩年就過去了,不知不覺也進入了尾聲。一路走來,獲得許多人的幫助,同時我也成長許多,度過了充實的碩士生涯。

感謝李宏毅教授,在兩年前願意讓我跟著教授學習,那時的我對語音辨識、機器學習等等技術完全一竅不通,甚至程式能力也不是特別精通。教授仍不嫌棄一步步的指導著我。在面對各種問題的時候,教授總是親切的跟我解答,提供我許多意見。教授即使再忙碌,也能夠撥出時間每週跟我開會。十分感謝教授兩年的指導。

感謝李琳山教授,提供了舒適的研究環境,且即使我並非老師的學生,但老師仍給予許多的建議與幫助。在每周的例行開會,從老師的身上吸取了許多知識。且語音實驗室傳統的活動,使我快速融入了語音實驗室的大家庭,與大家一起快樂研究。

感謝呂相弘學長,不論在學術討論跟玩樂上都帶領著我。感謝廖宜修學長,當我在工作站遇到問題,甚至程式上的問題,都熱心的幫我解決,是個強大的網管。感謝沈昇勳學長跟曾柏翔學長,和你們一起看球賽,打籃球,路上到處吃冰淇淋,格外開心,同時在研究進度上也提供我許多意見。感謝許家興學長,提供我研究的方向與資料,並且加油打氣。感謝吳彥諶學長,雖然學長的時區有點難抓,不過和學長一起打打電動,度過無數個快樂夜晚。再次感謝531語音實驗室的學長們,你們的能力都很強,對研究非常有熱忱,是我的榜樣。

感謝吳柏瑜、沈家豪、王育軒、陳永哲、林賢進、陳仰德、劉家翔、余朗祺,在修課上面,幫助了我許多,在研究上,也提供我很多建議與幫助,與你們一同度過兩年的時光,十分快樂,感謝有你們。感謝531語音實驗室的學弟妹,有你們打理實驗室的一切,且偶而還能聽聽你們的趣事,十分有趣,未來一年的日子,祝你們順順利利。

感謝我的家人,辛苦拉拔我長大,也給我許多的自由,對我的選擇全力支持,雖然我每週跟您們見不到幾次面,但一見面就是滿滿的關心,讓我感受到家的溫暖。感謝我的女朋友,在我忙碌時陪伴著我,甚至自己來台大找我一起吃晚餐,有你的陪伴,使我能夠穩穩的走完這兩年,未來你就是正式老師了,要一起加油。

感謝張家瑋、陳永哲、李洪瑞、鍾佩宏、洪瑄蔓、謝府諺、王維浩、潘彥宇、孫也喨,這些大學同學們,在台大一起游泳,一起運動,一起吃飯,一起適應這新的環境,祝福你們畢業順利。

最後感謝我自己,選擇了繼續讀碩士,選擇了臺大,選擇了語音實驗室,才能夠讓我兩年中有些成長,能夠讓我進入更高的知識殿堂,認識了許多實力堅強的學長姐,希望未來的我,也能夠朝著自己的路順利前進。
\end{acknowledgement}

\begin{zhAbstract}  %中文摘要
本論文之主軸在探討語音數位內容之口述詞彙偵測。由於近年來網路蓬勃發展,使得網路上包含語音資訊的多媒體如線上課程、電影、戲劇、會議錄音等日漸增加,因此,語音數位內容之檢索也隨之受到重視。語音數位內容檢索的關鍵部分為口述語彙偵測,找出語音文件中出現查詢詞的部分。本論文的查詢詞為語音訊號,並非文字。傳統的方法都會藉由語音辨識系統先將查詢詞轉為文字,而本論文則不經過語音辨識系統,使用機器學習中的類神經網路,在訓練語料中學習聲音的特徵,如此便可直接在語音訊號上進行口述詞彙偵測,以避免語音辨識系統錯誤率影響檢索系統的問題。

本論文採用了專注式機制,此機制能夠使模型關注在語音文件中某個區塊,避免多餘的雜訊影響。回顧機制能夠使模型依照先前的輸入而關注在語音文件中不同地方,進而模型能夠多次關注語音文件,且更精準的找到查詢詞。同時也嘗試使用語音詞向量,將語音文件編碼成為一向量,其向量能夠有詞與詞之間的相關性,藉由語音文件向量進行口述詞彙偵測。
\end{zhAbstract}

{
%\zhKaiFont
\mysinglespacing\selectfont
\tableofcontents %目錄

\listoffigures  %圖目錄

\listoftables  %表目錄
\par
}

\newpage
\setcounter{page}{1}
\pagenumbering{arabic}
