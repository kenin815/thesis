\NTUtitlepage  % 產生論文封面

\newpage
\setcounter{page}{1}
\pagenumbering{roman}

\NTUoralpage  % 產生口試委員會審定書

\mydoublespacing
\begin{acknowledgement} %誌謝
轉眼之間,碩士兩年就過去了。
\end{acknowledgement}

\begin{zhAbstract}  %中文摘要
本論文之主軸在探討語音數位內容之口述詞彙偵測。由於近年來網路蓬勃發展,使得網路上包含語音資訊的多媒體如線上課程、電影、戲劇、會議錄音等日漸增加,因此,語音數位內容之檢索也隨之受到重視。語音數位內容檢索的關鍵部分為口述語彙偵測,找出語音文件中出現查詢詞的部分。本論文
不經過語音辨識系統,使用機器學習中的類神經網路,在訓練語料中學習聲音的特徵,如此便可直接在語音訊號上進行口述詞彙偵測,以避免語音辨識系統錯誤率影響檢索系統的問題,也能夠避免經過語音辨識後,喪失掉許多語音中珍貴的資訊。

本論文將採用了專注式機制使模型能夠關注在語音文件的一部分,避免多餘的雜訊影響,以及回顧機制使模型依照之前的輸入關注在不同的地方。同時也使用了語音詞向量的概念,訓練出語音文件向量進行口述詞彙偵測。
\end{zhAbstract}

{
%\zhKaiFont
\mysinglespacing\selectfont
\tableofcontents %目錄

\listoffigures  %圖目錄

\listoftables  %表目錄
\par
}

\newpage
\setcounter{page}{1}
\pagenumbering{arabic}
